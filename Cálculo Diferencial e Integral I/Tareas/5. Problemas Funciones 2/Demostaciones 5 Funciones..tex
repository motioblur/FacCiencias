\documentclass[11pt, a4paper]{article}
\usepackage[utf8]{inputenc}
\usepackage{amsmath}
\usepackage{amssymb}
\usepackage{amsthm}
\usepackage{parskip}
\usepackage{enumitem}

\title{Cálculo diferencial e integral I\\Resolución de Problemas de Funciones 2}
\author{Vite Riveros Carlos Emilio}
\date{19 septiembre del 2022}

\begin{document}
    \maketitle
    2. Encuentre $f^{-1}$ si:
    \begin{center}
        iv. $f(x)=\{x \text{ si $x$ es racional}, -x \text{ si $x$ es irracional}\} $

        v. $f(x)=\{-x^2 \text{ si $x\geq 0$}, 1-x^3 \text{ si $x<0$}\}$
    \end{center}
    3. Sean $f$ y $g$ dos funciones:
    \begin{center}
        \begin{enumerate}[label= \roman*]
            \item Pruebe que, si $f$ es inyectiva y $g(x)=f(x)+1$, entonces $g$ es inyectiva. Exprese a $g^{-1}$ en 
            términos de $f^{-1}$.

            \item Pruebe que, si $Ima(g) \subset dom(f)$ y $f$ y $g$ son inyectivas, entonces $f \circ g$ es inyectiva. 
            Exprese a $(f \circ g)^{-1}$ en términos de $f^{-1}$ y $g^{-1}$.

        \end{enumerate}
    \end{center}
    4. Pruebe que, si $A, B \in \mathbb{R}$, entonces existen $a,b \in \mathbb{R}$ tales que:
    \begin{center}
        $A\sin{(x+B)}=a\sin{(x)}+b\cos{(x)}$
    \end{center}
    8. Pruebe que:
    \begin{center}
        $\sin^2(x)=\frac{1}{2}(1-\cos (2x))$
    \end{center}
    Para cualesquiera $x \in \mathbb{R}$. Deduzca una fórmula análoga para $cos^2 (x)$.
\end{document}