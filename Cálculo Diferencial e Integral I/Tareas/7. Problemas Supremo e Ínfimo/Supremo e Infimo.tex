\documentclass[11pt, a4paper]{article}
\usepackage[utf8]{inputenc}
\usepackage{amsmath}
\usepackage{amssymb}
\usepackage{amsthm}
\usepackage{parskip}

\title{Cálculo diferencial e integral I\\Resolución de Problemas de Supremo e Ínfimo}
\author{Vite Riveros Carlos Emilio}
\date{20 octubre del 2022}


\begin{document}
    \maketitle
    \begin{enumerate}
        \item Sea $S \subseteq \mathbb{R}$ un conjunto no vacío. Demostrar que $u \in \mathbb{R}$ es una cota superior de $S$ si y sólo si las
        condiciones: $t \in \mathbb{R}$ y $t > u$, implican que $t \notin S$.

        \begin{proof}
            Primero demostraremos que $t \notin S$. Sabemos que $t > u$ y que por hipótesis $u$ es cota superior de $S$.
            Por definición de cota superior, sabemos que $\forall x \in S$, $x \leq u$, y como $u<t$, tenemos entonces:
            \begin{center}
                $\forall x \in S$, $x\leq u < t$

                $\forall x \in S$, $x < t$
            \end{center}

            Por lo que concluimos que $t$  es una cota superior y que $t \notin S$ al ser estrictamente mayor que cualesquiera elemento de $S$.

            Ahora demostraremos que $u$ es cota superior de $S$, sabiendo que $t>u$. Por hipótesis, sabemos que $t \notin S$.


        \end{proof}

        \item Sea $S$ un subconjunto no vacío de $\mathbb{R}$ que está acotado inferiormente. Demostrar que el conjunto
        $-S = \{-s | s \in S\}$ está acotado superiormente y que:
        \begin{center}
            $\inf(S)=-\sup(-S)$
        \end{center}

        \item (6.)  Sean $S$ y $S_0$ subconjuntos no vacíos de $\mathbb{R}$ tales que $S_0 \subseteq S \subseteq \mathbb{R}$ y $S$ acotado superior e inferiormente.
        Demostrar que:

        \begin{center}
            $\inf(S)\leq\inf(S_0)\leq\sup(S_0)\leq\sup(S)$
        \end{center}

        \item (8.) Si $S = \{\frac{1}{n}-\frac{1}{m}|:n,m \in \mathbb{N}\}$, demuestre que $S$ está acotado superior e inferiormente y encuentre $\inf(S)$
        y $\sup(S)$. Pruebe su respuesta.
    \end{enumerate}
\end{document}