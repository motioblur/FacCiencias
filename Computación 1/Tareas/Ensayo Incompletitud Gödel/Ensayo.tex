\documentclass{article}
\usepackage[utf8]{inputenc}
\usepackage{parskip}

\title{Miniensayo sobre la Máquina de Turing}
\author{Vite Riveros Carlos Emilio, Carrera: Física}
\date{31 de agosto}

\begin{document}
    \maketitle

    El concepto de la máquina de Turing es, sino el más importante, uno de los más importantes de 
    las ciencias de la computación. Y este a su vez tiene su origen en una serie de problemas
    planteados a inicios del siglo XX que cuestionaban en sí, la propia naturaleza de las matemáticas. 
    

    Para entender qué es una máquina de Turing y porque todos los sistemas de computo modernos, 
    como la estructura de Von Neumann, están basados en esta, necesitaremos entender estos problemas.


    Los problemas fueron planteados por David Hilbert, como parte de su lista de los 23 problemas más 
    importantes de las matemáticas, escrita como una lista de retos para los matemáticos del nuevo siglo
    y que expresó en forma de preguntas. Entre ellos vienen muchos problemas de gran relevancia pero los que 
    tuvieron quizá el mayor impacto y los que nos interesan son los siguientes:

    \begin{enumerate}
        \centering
        \item ¿Las matemáticas son completas?\\
        Esto se refiere a que si es posible probar o refutar cualquier enunciado matemático partiendo de 
        los axiomas.
    
        \item ¿Las matemáticas son consistentes?\\
        Se refiere a si solo se pueden probar los problemas verdaderos, ya que un sistema donde se pueden 
        probar enunciados falsos es un sistema inconsistente e inconfiable.

        \item ¿Son todos los enunciados de las matemáticas decidibles?\\
        Esto se refiere a que si existe un procedimiento que puede ser aplicado para cualquier enunciado que 
        nos diría si este es verdadero o falso en tiempo finito.
    \end{enumerate}
    
    Hilbert aseguraba que se encontraría la respuesta a estos problemas, y no solo eso, sino que la respuesta sería
    que sí. Decía que no hay problemas inresolvibles en la matemática, de manera optimista.

    Sin embargo, un lógico austriaco de nombre Kurt Gödel demostraría lo contario. Ya que un día antes de que hiciera 
    esta famosa afirmación, Gödel demostró que la respuesta a la primera pregunta de Hilbert era que las matemáticas
    no podían ser un sistema completo. Terminando así con el sueño del formalista.

    La manera en la que Gödel demostró esto fue con un sistema que creo en el cual la matemática podría auto-referenciar 
    sus propias características. En este sistema, logró llegar a un enunciado con una finalidad simple. Este en palabras
    simples dice: 
    \begin{center}

    "Este enunciado no se puede demostrar bajo los axiomas". 

    \end{center}
    La finalidad de esto es llegar a un caso en el cual un enunciado tiene que ser verdadero, ya que si es falso, este probaría que 
    las matemáticas son inconsistentes, lo cual significaría que no son confiables como sistema de verdades. Pero la consecuencia de esto 
    es que obtienes un enunciado que tiene que ser verdad, pero no puedes probar. Asimismo, este enunciado en sí no puede ser tratado como 
    un axioma más, ya que eso causaría una cadena de enunciados que tendrían que ser axiomáticos también.

    Con esto, Gödel probó que las matemáticas no eran completas, y puso en duda su consistencia. Pero el último problema es el que nos lleva al 
    origen de la computacion moderna. Para 1935, el británico Alan Turing tenía la respuesta, y una vez más era un no. Las matemáticas no eran
    decidibles.

    Para descubrir la respuesta de esta última pregunta, Turing se basó en la noción de Leibniz; en la cual una máquina lo suficientemente poderosa
    para poder computar lo que sea podría calcular con certeza cualquier operación en tiempo finito. Esto hizo que Turing ideara una máquina teórica 
    que cumpliera con estas propiedades, la máquina de Turing. Esta idea es el concepto fundacional del que se desprende la computación electrónica 
    programable de la modernidad.

    Una máquina de Turing, en términos simples, es una máquina que consiste en tres partes. Una cinta infinita en la que puede escribir y leer información
    con una cabeza, que a su vez, la misma cabeza que lee y escribe información. Y una serie de reglas que le dicen a la cabeza a qué celda de la cinta debe 
    moverse, o en su defecto si debe detenerse.

    Con esta máquina puede demostrarse el razonamiento de Turing. Si se asume que existe una máquina de Turing llamada H que es capaz de determinar si cualquier
    programación que le sea insertada terminará en un ciclo infinito o se detendrá en un tiempo finito sin falla alguna. Si uno toma H e invierte su salida, es decir, que si H 
    da como resultado que se detiene la operación, H comienza un ciclo infinito y viceversa. Y después se le proporciona a H con el mismo programa de H con esta
    inversión, H es incapaz de llegar a una conclusión determinada, por lo tanto su H nunca fue una máquina que acertara siempre, entonces H no existe. Con esto Alan 
    Turing probó que no es posible demostrar que para cualquier enunciado es posible determinar su veracidad en un tiempo finito.

    Pero Turing, al intentar probar que este último enunciado era falso, terminó creando las bases conceptuales de lo que sería la computación electrónica programable.
    Y 10 años después esto sirvió para la creación de máquinas que eran capaces de descifrar el código nazi Enigma, a lo que se le atribuye el recortar
    el tiempo que duró la guerra. Todo esto para eventualmente llegar a la creación de la primera computadora programable de uso general: ENIAC. Hasta que John Von Neumann
    planteó la estructura en la que las computadoras actuales están realmente basadas.

    Esta consiste de una unidad de procesamiento con una unidad aritmética lógica y registros de procesos, una unidad de control que incluye un registro de 
    instrucciones y un contador de programa, memoria que pueda almacenar información e instrucciones, una memoria externa en masa y mecanismos de entrada y sálida.
    En conclusión, los problemas de Hilbert de manera inadvertida llevaron a la humanida a la invención de una de las herramientas más poderosas de su existencia, ya que
    las computadoras actuales no son conceptualmente diferentes de lo planteado por Von Neuman en los 40.



    \textbf{Bibliografía}\\
    Mitchell, M. (2009). Complexity a Guide Tour. New York: Oxford Univeristy Press.\\
    Numberphile. (31 de mayo de 2017). Gödel's Incompleteness Theorem - Numberphile. Reino Unido. Obtenido de https://www.youtube.com/watch?v=O4ndIDcDSGc\\
    Veritasium. (22 de mayo de 2021). Math's Fundamental Flaw. Estados Unidos. Obtenido de https://www.youtube.com/watch?v=HeQX2HjkcNo


\end{document}