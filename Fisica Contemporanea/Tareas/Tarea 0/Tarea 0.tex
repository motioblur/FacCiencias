\documentclass[10pt, a4paper]{article}
\usepackage[utf8]{inputenc}
\usepackage{amsmath}
\usepackage{amssymb}
\usepackage{amsthm}
\usepackage{parskip}
\usepackage{siunitx}

\title{Física Contemporánea\\Resolución de Tarea 0}
\author{Vite Riveros Carlos Emilio}
\date{3 septiembre del 2022}

\begin{document}
    \maketitle

    \begin{enumerate}
        \item \textbf{Problemas}
            \begin{itemize}
                \item \textbf{Unidades} \\
                Dadas las constantes de la Naturaleza en MKS, a saber la velocidad de la luz en el vacío, 
                $c = 2.99792 * 10^8 \frac{\si{m}}{\si{s}}$, la constante de la gravitacion,
                $G = 6.6738 * 10^{-11} \frac{\si{m^3}}{\si{kg.s^2}}$ y la constante de Planck $h = 6.6260 * 10^{-34} \frac{\si{kg.m^2}}{\si{s}}$, 
                haz combinaciones entre ellas para determinar una cantidad que tenga unidades
                de distancia (metros), otra que tenga unidades de tiempo (segundos) y otra que
                tenga unidades de masa (Kg). Dichas cantidades se conocen como la distancia
                de Planck, el tiempo de Planck y la masa de Planck. Discute sobre su posible
                significado.
                \begin{center}
            
                    $[c]^{a}[G]^{b}[h]^{c}=[\frac{\si{m}}{\si{s}}]^a[\frac{\si{m^3}}{\si{kg.s^2}}]^b[\frac{\si{kg.m^2}}{\si{s}}]^c$

                    $[c]^{a}[G]^{b}[h]^{c}=(\si{m.s^{-1}})^a(\si{m^3.kg^{-1}.s^{-2}})^b(\si{kg.m^2.s^{-1}})^c$

                    $[c]^{a}[G]^{b}[h]^{c}=\si{m^a.s^{-a}.m^{3b}.kg^{-b}.s^{-2b}.kg^c.m^{2c}.s^{-c}}$

                    $[c]^{a}[G]^{b}[h]^{c}=\si{(m^a.m^{3b}.m^{2c})(s^{-a}.s^{-2b}.s^{-c})(kg^{-b}.kg^c)}$

                    $[c]^{a}[G]^{b}[h]^{c}=\si{m^{a+3b+2c}.s^{-a-2b-c}.kg^{-b+c}}$
                
                \end{center}
                
                Esto nos da el sistema de ecuaciones:

                \begin{enumerate}
                    \item $a+3b+2c=0$. Para la potencia de $\si{m}$
                    \item $-a-2b-c=0$. Para la potencia de $\si{s}$
                    \item $-b+c=0$. Para la potencia de $\si{kg}$
                \end{enumerate}

                Para despejar $[c]^{a}[G]^{b}[h]^{c}=\si{m}$:
                \begin{center}
                
                    $a+3b+2c=1$, $-a-2b-c=0$

                    $-b+c=0$, $c=b$

                    $a+3c+2c=1$, $a+5c=1$

                    $-a-2c-c=0$, $-a-3c=0$
                    
                    $(-a-3c=0)+(a+5c=1)$, $2c=1$
                    
                    $c=\frac{1}{2}=b$

                    $a+5(\frac{1}{2})=1$, $a+\frac{5}{2}=1$

                    $a=-\frac{3}{2}$
                \end{center}

                Después sustituimos en las potencias:
                \begin{center}
                    $[c]^{-\frac{3}{2}}[G]^{\frac{1}{2}}[h]^{\frac{1}{2}}=\si{m}$

                    $=(2.99792 * 10^8)^{-\frac{3}{2}}(6.6738 * 10^{-11})^{\frac{1}{2}}(6.6260 * 10^{-34})^{\frac{1}{2}}$

                    $=(2.99792^{-\frac{3}{2}} * 10^{8(-\frac{3}{2})})(6.6738^{\frac{1}{2}} * 10^{-11(\frac{1}{2})})(6.6260^{\frac{1}{2}} * 10^{-34(\frac{1}{2})})$

                    $=\frac{1}{\sqrt[]{2.99792^3}} * 10^{-12} * \sqrt[]{6.6738} * 10^{-\frac{1}{2}} * 10^{-\frac{10}{2}} * \sqrt[]{6.6260} * 10^{-17}$

                    $=\frac{\sqrt[]{6.6738}(\sqrt[]{6.6260})}{\sqrt[]{2.99792^3}(\sqrt[]{10})}*10^{-12}*10^{-5}*10^{-17}$
                    
                    $=\sqrt[]{\frac{6.6738(6.6260)}{2.99792^3(10)}}*10^{-34}$

                    $=0.4051*10^{-34}$ $\si{m}$

                    $=4.051*10^{-35}$ $\si{m}$
                \end{center}

                Esta distancia se llama la distancia de Planck, y puede ser la distancia mínima en la que 
                podemos usar nuestro conocimiento de la física para predecir y explicar la realidad.

                Para despejar $[c]^{a}[G]^{b}[h]^{c}=\si{s}$:
                \begin{center}
                
                    $a+3b+2c=0$, $-a-2b-c=1$

                    $-b+c=0$, $c=b$

                    $a+3b+2b=0$, $a+5b=0$

                    $-a-2b-b=1$, $-a-3b=1$
                    
                    $(-a-3b=1)+(a+5b=0)$, $2b=1$
                    
                    $c=\frac{1}{2}=b$

                    $a+5(\frac{1}{2})=0$, $a+\frac{5}{2}=0$

                    $a=-\frac{5}{2}$
                \end{center}

                Después sustituimos en las potencias:
                \begin{center}
                    $[c]^{-\frac{5}{2}}[G]^{\frac{1}{2}}[h]^{\frac{1}{2}}=\si{s}$

                    $=(2.99792 * 10^8)^{-\frac{5}{2}}(6.6738 * 10^{-11})^{\frac{1}{2}}(6.6260 * 10^{-34})^{\frac{1}{2}}$

                    $=(2.99792^{-\frac{5}{2}} * 10^{8(-\frac{5}{2})})(6.6738^{\frac{1}{2}} * 10^{-11(\frac{1}{2})})(6.6260^{\frac{1}{2}} * 10^{-34(\frac{1}{2})})$

                    $=\frac{1}{\sqrt[]{2.99792^5}} * 10^{-20} * \sqrt[]{6.6738} * 10^{-\frac{1}{2}} * 10^{-\frac{10}{2}} * \sqrt[]{6.6260} * 10^{-17}$

                    $=\frac{\sqrt[]{6.6738}(\sqrt[]{6.6260})}{\sqrt[]{2.99792^5}(\sqrt[]{10})}*10^{-20}*10^{-5}*10^{-17}$
                    
                    $=\sqrt[]{\frac{6.6738(6.6260)}{2.99792^5(10)}}*10^{-42}$

                    $=.1351*10^{-42}$ $\si{s}$

                    $=1.351*10^{-43}$ $\si{s}$
                \end{center}
                
                Este tiempo llamado el tiempo de Planck, puede significar el tiempo mínimo en el que podemos 
                detectar un fenomeno, y quizá puede decirnos algo al respecto de la posible cuantización del 
                espacio-tiempo junto con la distancia de Planck.

                Para despejar $[c]^{a}[G]^{b}[h]^{c}=\si{kg}$:
                \begin{center}
                
                    $a+3b+2c=0$, $-a-2b-c=0$, $-b+c=1$

                    $(a+3b+2c=0)+(-a-2b-c=0)=(b+c=0)$, $c=-b$

                    $-b+c=1$, $-2b=1$, $b=-\frac{1}{2}$, $c=\frac{1}{2}$

                    $-a-2(-\frac{1}{2})-\frac{1}{2}=0$, $-a+1-\frac{1}{2}=0$

                    $-a+\frac{1}{2}=0$, $a=\frac{1}{2}$
                \end{center}

                Después sustituimos en las potencias:
                \begin{center}
                    $[c]^{\frac{1}{2}}[G]^{-\frac{1}{2}}[h]^{\frac{1}{2}}=\si{s}$

                    $=(2.99792 * 10^8)^{\frac{1}{2}}(6.6738 * 10^{-11})^{-\frac{1}{2}}(6.6260 * 10^{-34})^{\frac{1}{2}}$

                    $=(2.99792^{\frac{1}{2}} * 10^{8(\frac{1}{2})})(6.6738^{-\frac{1}{2}} * 10^{-11(-\frac{1}{2})})(6.6260^{\frac{1}{2}} * 10^{-34(\frac{1}{2})})$

                    $=\frac{1}{\sqrt[]{2.99792}} * 10^4 * \sqrt[]{6.6738} * 10^{-\frac{1}{2}} * 10^5 * \sqrt[]{6.6260} * 10^{-17}$

                    $=\frac{\sqrt[]{6.6738}(\sqrt[]{6.6260})(\sqrt[]{10})}{\sqrt[]{2.99792}}*10^4*10^5*10^{-17}$
                    
                    $=\sqrt[]{\frac{6.6738(6.6260)(10)}{2.99792^5}}*10^{-8}$

                    $=5.4557*10^{-8}$ $\si{s}$
                \end{center}

                Esta masa se llama la masa de Planck y puede significar el mínimo de lo que podemos medir 
                del cambio de la masa de una partícula.

                \item \textbf{Derivadas} \\
                Usando que $u(x) = \sqrt[]{5x^2-2x+9}$ y que $v(x) = 7\cos(2x^3 + 8)$, 
                determina las derivadas de $f = uv$ y de $g = \frac{u}{v}$

                \begin{center}
                    $u(x)=\sqrt[]{5x^2-2x+9}$, $a(b)=\sqrt[]{b}$, $b(x)=5x^2-2x+9$

                    $\frac{du}{dx}u(x)=\frac{da}{db}\frac{db}{dx}$

                    $u'(x)=\frac{1}{2(\sqrt[]{a})}(10x-2)$

                    $u'(x)=\frac{10x-2}{2(\sqrt[]{5x^2-2x+9})}$

                    $v(x) = 7\cos(2x^3 + 8)$, $c(l)=\cos(l)$, $l(x)=2x^3 + 8$

                    $\frac{dv}{dx}v(x)=7(\frac{dc}{dl}\frac{dl}{dx})$

                    $v'(x)=7(-\sin(l))(6x^2)$

                    $v'(x)=-42x^2\sin(2x^3 + 8)$

                    $f = uv$, $f'(x)= \frac{du}{dx}(v) + \frac{dv}{dx}(u)$

                    $f'(x)=(\frac{10x-2}{2(\sqrt[]{5x^2-2x+9})})(7\cos(2x^3 + 8))+(-42x^2\sin(2x^3 + 8))(\sqrt[]{5x^2-2x+9})$

                    $f'(x)=\frac{(70x-14)\cos(2x^3 + 8)}{2(\sqrt[]{5x^2-2x+9})}-42x^2\sin(2x^3 + 8)(\sqrt[]{5x^2-2x+9})$

                    $g=\frac{u}{v}$, $g(x)=(\frac{1}{7})(\sqrt[]{5x^2-2x+9})(\sec(2x^3 + 8))$

                    $g'(x)=\frac{1}{7}(\frac{du}{dx})v(x)+\frac{1}{7}(\frac{dv}{dx})u(x)$

                    $g'(x)=\frac{1}{7}(\frac{10x-2}{2(\sqrt[]{5x^2-2x+9})})(\sec(2x^3 + 8))+\frac{1}{7}(6x^2\sec(2x^3 + 8)\tan(2x^3 + 8))(\sqrt[]{5x^2-2x+9})$

                    $=\frac{(10x-2)\sec(2x^3 + 8)}{14(\sqrt[]{5x^2-2x+9})}+\frac{6x^2(\sqrt[]{5x^2-2x+9})\sec(2x^3 + 8)\tan(2x^3 + 8)}{7}$
                    
                \end{center}

                \item \textbf{Más derivadas} \\
                Deriva por favor las siguientes funciones, considerando que
                A, B, C, $\omega$, $\delta$, $\sigma$ son constantes:

                \begin{itemize}
                    \item$f(t) = A\cos(\omega t + \delta)$\\
                    $f'(t)=A(-\sin(\omega t + \delta))(\omega)$\\
                    $f'(t)=-A\omega \sin(\omega t + \delta)$

                    \item$f(t) = A\sin(\omega t + \delta)$\\
                    $f'(t)=A(\cos(\omega t + \delta))(\omega)$\\
                    $f'(t)=A\omega \cos(\omega t + \delta)$

                    \item$g(x) = \sqrt[]{B\ln(x^2+\sigma)}$\\
                    $g'(x) = \frac{1}{(2)\sqrt[]{B \ln(x^2+\sigma)}}(\frac{B}{x^2+\sigma}(2x))$\\
                    $g'(x) = \frac{1}{(2)\sqrt[]{B\ln(x^2+\sigma)}}(\frac{2Bx}{x^2+\sigma})$\\
                    $g'(x) = \frac{2Bx}{x^2+\sigma(2)(\sqrt[]{B\ln(x^2+\sigma)})}$\\
                    $g'(x) = \frac{Bx}{x^2+\sigma(\sqrt[]{B\ln(x^2+\sigma)})}$

                    \item$h(y) = A e^{By^2+\delta}$\\
                    $h'(y) = A e^{By^2+\delta}(2By)$\\
                    $h'(y) = 2ABy e^{By^2+\delta}$

                    \item$y(x) = Ax^2$\\
                    $y'(x) = 2Ax$
                \end{itemize}

                Repite el cálculo (o substituye) para cuando $A = -3$, $B = 5$, $\omega = 2\pi$, $\delta = \frac{\pi}{2}$,
                $\sigma = 7$ y finalmente, evalúa a las derivadas cuando la variable es igual a cero y a uno.

                \begin{itemize}
                    \item$f(t) = A\cos(\omega t + \delta)$\\
                    $f'(t)=A(-\sin(\omega t + \delta))(\omega)$\\
                    $f'(t)=6\pi \sin(2\pi t + \frac{\pi}{2})$

                    $f'(0)=6\pi \sin(\frac{\pi}{2})$
                    $f'(0)=0.1722$

                    $f'(1)=6\pi \sin(2\pi + \frac{\pi}{2})$
                    $f'(1)=2.5757$
                    

                    \item$f(t) = A\sin(\omega t + \delta)$\\
                    $f'(t)=A(\cos(\omega t + \delta))(\omega)$\\
                    $f'(t)=-6\pi \cos(2\pi t + \frac{\pi}{2})$

                    $f'(0)=-6\pi \cos(\frac{\pi}{2})$
                    $f'(0)=-0.1722$

                    $f'(1)=-6\pi \sin(2\pi + \frac{\pi}{2})$
                    $f'(1)=-2.5757$

                    \item$g(x) = \sqrt[]{B\ln(x^2+\sigma)}$\\
                    $g'(x) = \frac{1}{(2)\sqrt[]{B \ln(x^2+\sigma)}}(\frac{B}{x^2+\sigma}(2x))$\\
                    $g'(x) = \frac{1}{(2)\sqrt[]{B\ln(x^2+\sigma)}}(\frac{2Bx}{x^2+\sigma})$\\
                    $g'(x) = \frac{2Bx}{x^2+\sigma(2)(\sqrt[]{B\ln(x^2+\sigma)})}$\\
                    $g'(x) = \frac{5x}{x^2+7(\sqrt[]{5\ln(x^2+7)})}$

                    $g'(0) = \frac{0}{7(\sqrt[]{5\ln(7)})}$
                    $g'(0) = 0$

                    $g'(1) = \frac{5}{8(\sqrt[]{5\ln(8)})}$
                    $g'(1) = \frac{5}{8(\sqrt[]{5\ln(8)})}$
                    $g'(1) = 0.1938$

                    \item$h(y) = A e^{By^2+\delta}$\\
                    $h'(y) = A e^{By^2+\delta}(2By)$\\
                    $h'(y) = -30y e^{5y^2+\frac{\pi}{2}}$

                    $h'(0) = -30(0) e^{5y^2+\frac{\pi}{2}}$
                    $h'(y) = 0$

                    $h'(1) = -30 e^{5+\frac{\pi}{2}}$
                    $h'(1) = -21418.1443$

                    \item$y(x) = Ax^2$\\
                    $y'(x) = -6x$

                    $y'(0) = 0$

                    $y'(1) = -6$

                \end{itemize}

                \item \textbf{Integrales} \\
                Así mismo, integra:
                
                \begin{center}
                    $\int_{A}^{B}(Ct+\delta)dt$\\
                    $=\frac{Ct^2}{2}(\delta t)$

                    $\int_{A}^{B}(\sigma t^3+\omega t^2 + Ct + \delta)dt$\\
                    $=\frac{Ct^2}{2}+\frac{\sigma^3t^4}{4}+\frac{t^3\omega^2}{3}+\delta t + c$

                    $\int_{3}^{8}(-2t^3+7t^2-t+9)dt$\\
                    $=-\frac{t^4}{2}+\frac{7t^3}{3}+2t+c$

                    $\int_{}^{}A\cos(x)dx$\\
                    $=A\sin(x) + c$

                    $\int_{}^{}B\sin(x)dx$\\
                    $=-A\cos(x) + c$

                    $\int_{0}^{\pi}\cos(x)dx$\\
                    $=A\sin(x) + c$

                    $\int_{0}^{\pi}2\sin(x)dx$\\
                    $=-2\cos(x) + c$
                \end{center}
            \end{itemize}
            
        \item \textbf{Preguntas}
            \begin{itemize}   
                \item ¿Cuáles son tus motivaciones para estudiar física?\\
                Tengo intereses diversos, y creo que la física abarca muchos de ellos.

                \item Describe tus estrategias de estudio, las horas que piensas que le debes dedicar a
                las tareas y las que le dedicas\\
                Suelo leer el tema hasta entenderlo y practicar problemas o lo que sea necesario hasta que
                pueda manejarlo con facilidad. Actualmente le dedico la mayor parte de todos mis días a estudiar
                y planeo seguir haciéndolo.

                \item Describe dos temas de Física sobre los que te gustaría conocer más.\\
                Me gustaría conocer acerca de la física computacional y la física nuclear.
            \end{itemize}
    \end{enumerate}
\end{document}