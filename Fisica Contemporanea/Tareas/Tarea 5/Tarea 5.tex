\documentclass[10pt, a4paper]{article}
\usepackage[utf8]{inputenc}
\usepackage{amsmath}
\usepackage{amssymb}
\usepackage{amsthm}
\usepackage{parskip}
\usepackage{enumitem}
\usepackage{siunitx}
\usepackage{tikz}
\usepackage{pgfplots}
\usepgfplotslibrary{polar}
\usepackage{graphicx}
\graphicspath{ {./} }
\usetikzlibrary{arrows.meta}
\usetikzlibrary{angles,quotes}
\pgfplotsset{compat = newest}

\title{Física Contemporánea\\Resolución de Tarea 5}
\author{Vite Riveros Carlos Emilio}
\date{21 de noviembre del 2022}

\begin{document}
    \maketitle

    \begin{enumerate}
        \item Dos cargas puntuales $+Q$ y $-Q$ separadas una distancia $d$ (dipolo eléctrico)
        están en el eje $x$, en $x = \frac{d}{2}$ y $x=-\frac{d}{2}$ respectivamente. Encuentra 
        la fuerza neta sobre una tercera carga $+q$, también en el eje $x$ en $x <\frac{d}{2}$.
        Simplifica el resultado y obtén la forma de la fuerza neta aproximada para $x >> d$.

        \begin{center}
            $\vec{F_e}=q(\Sigma(\vec{E}))$\\
            $r_1 = |x-\frac{d}{2}|$, $r_2 = |x+\frac{d}{2}|$\\
            $\vec{F_e}=q(\frac{k(+Q)}{r_1^2}\vec{r}+\frac{k(-Q)}{r_2^2}\vec{r})$\\
            $\vec{F_e}=q(\frac{k(+Q)}{|x-\frac{d}{2}|^2}\vec{r}+\frac{k(-Q)}{|x+\frac{d}{2}|^2}\vec{r})$\\
            $\vec{F_e}=kq(\frac{+Q}{(x-\frac{d}{2})^2}\vec{r}+\frac{-Q}{(x+\frac{d}{2})^2}\vec{r})$\\
            $\vec{F_e}=kq(\frac{+Q}{(x-\frac{d}{2})^2}+\frac{-Q}{(x+\frac{d}{2})^2})(1, 0)$

            $r_1 = \lim_{x\to d}|x-\frac{d}{2}|=|d-\frac{d}{2}|=\frac{d}{2}$\\
            $r_2 = \lim_{x\to d}|x+\frac{d}{2}|=|d+\frac{d}{2}|=\frac{3d}{2}$\\
            $\vec{F_e}=kq(\frac{+Q}{(\frac{d}{2})^2}+\frac{-Q}{(\frac{3d}{2})^2})(1, 0)$\\
            $\vec{F_e}=kq(\frac{+Q}{\frac{d^2}{4}}+\frac{-Q}{\frac{9d^2}{4}})(1, 0)$\\
            $\vec{F_e}=kq(\frac{4+Q}{d^2}+\frac{4-Q}{9d^2})(1, 0)$\\
            $\vec{F_e}=\frac{4kq}{d^2}(+Q+\frac{-Q}{9})(1, 0)$\\

        \end{center}

        \item Una barra delgada de longitud $L$ se coloca sobre el eje $x$. Una carga puntual
        $q$ se coloca sobre el mismo eje $x$ a una distancia $d$ de uno de los extremos de
        la barra. La barra tiene una distribución uniforme de carga, de $n$ Coulombs
        por metro. Calcula la fuerza eléctrica que actúa sobre la carga $q$.
        Sugerencia. Suma las contribuciones a la fuerza debidas a diferenciales de
        carga de la barra para obtener la fuerza total.

        \begin{center}
            $\lambda=\frac{Q_{\text{total}}}{L}$\\
            $dq = \lambda dL$\\
            $d\vec{E}=k\frac{dq}{x^2}$\\
            $d\vec{F_e}=q_a(d\vec{E})$\\
            $\int d\vec{F_e}=\int q_a(d\vec{E})$\\
            $\vec{F_e}= q_a\int{d\vec{E}}$\\
            $\vec{F_e}=q_a\int_{d}^{d+L}k\frac{dq}{x^2}$\\
            $\vec{F_e}=kq_a\int_{d}^{d+L}\frac{\lambda dx}{x^2}$\\
            $\vec{F_e}=\lambda kq_a\int_{d}^{d+L}{x^{-2}dx}$\\
            $\vec{F_e}=-\frac{\lambda kq_a}{x}_d^{d+L}$\\
            $\vec{F_e}=-[\frac{\lambda kq}{d+L}-\frac{\lambda kq}{d}]$\\
            $\vec{F_e}=\frac{\lambda kq}{d}-\frac{\lambda kq}{d+L}(1,0)$

        \end{center}

    \end{enumerate}
\end{document}
