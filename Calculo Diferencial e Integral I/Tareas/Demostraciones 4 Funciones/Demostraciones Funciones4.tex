\documentclass[11pt, a4paper]{article}
\usepackage[utf8]{inputenc}
\usepackage{amsmath}
\usepackage{amssymb}
\usepackage{amsthm}
\usepackage{parskip}

\title{Cálculo diferencial e integral I\\Resolución de Problemas de Funciones}
\author{Vite Riveros Carlos Emilio}
\date{9 septiembre del 2022}

\begin{document}
    \maketitle
    \begin{enumerate}
        \item Encuentre el dominio de las siguientes funciones:
        
        \begin{enumerate}
            \item $f(x)=\sqrt[]{1-x^2}$
            
            Para que $\sqrt[]{1-x^2}$ esté definida, se tiene que:
            \begin{center}
                $1-x^2\geq 0$

                $1\geq x^2$

                $\sqrt[]{1}\geq \sqrt[]{x^2}$

                $|x|\leq 1$

                $-1 \leq x \leq 1$

            \end{center}
            Entonces $dom f(x) = \{x \in \mathbb{R} | -1 \leq x \leq 1\}$.
            \item $f(x)=\sqrt[3]{1+x}$
            
            $\sqrt[3]{1+x}$ es una expresión de la forma $\sqrt[3]{a}$ por lo tanto su 
            dominio es $dom f(x)= \{x \in \mathbb{R}\}$.

            \item $f(x)=\sqrt[]{|1-x^2|}$
            
            Para que $\sqrt[]{|1-x^2|}$ esté definida, $|1-x^2| \geq 0$. Pero por definición
            del valor absoluto $|a|=a \geq 0$, entonces $dom f(x) = \{ x \in \mathbb{R}\}$.

            \item $f(x)=\frac{1}{1-x}+\frac{1}{x-2}$
            
            Para que $\frac{1}{1-x}+\frac{1}{x-2}$ esté definida se tiene que $1-x \neq 0$ y $x-2 \neq 0$, 
            entonces:
            \begin{center}
                $1-x = 0$, $x = 1$

                $x-2 = 0$, $x = 2$
            \end{center}

            Entonces cuando $x = 2$ o $x = 1$, no está definida la función, por lo que 
            $dom f(x)= \{x \in \mathbb{R} | x \neq 2, x \neq 1\}$.

            \item $f(x)=\sqrt[]{1-\sqrt[]{x^2-1}}$
            
            Para que $\sqrt[]{1-\sqrt[]{x^2-1}}$ esté definida, se tiene que $1-\sqrt[]{x^2-1} \geq 0$ y a su
            vez que $x^2-1 \geq 0$. Por lo que:

            \begin{center}
                $x^2-1 \geq 0$,  $x^2 \geq 1$

                $\sqrt[]{x^2} \geq \sqrt[]{1}$, $|x| \geq 1$

                $x \geq 1$, $x \leq -1$
            \end{center}

            Y ahora $1-\sqrt[]{x^2-1} \geq 0$:
            \begin{center}
                $1-\sqrt[]{x^2-1} \geq 0$

                $1 \geq \sqrt[]{x^2-1}$ 
                
                $1^2 \geq (\sqrt[]{x^2-1})^2$ 

                $1 \geq |x^2-1|$
                
                En el caso $x^2-1 \leq 1$:

                $x^2-1\leq 1$, $x^2 \leq 2$

                $\sqrt[]{x^2}\leq \sqrt[]{2}$, $|x|\leq \sqrt[]{2}$

                $x \leq \sqrt[]{2}$, $x \geq - \sqrt[]{2}$

                En el caso $x^2-1 \geq -1$:

                $x^2-1 \geq -1$, $x^2\geq 0$

                $\sqrt[]{x^2} \geq \sqrt[]{0}$, $|x|\geq 0$

                $x \geq 0$, $x \leq 0$

                Como en el segundo caso $x$ podía ser cualquier número en $\mathbb{R}$. Tenemos que
                $x \leq \sqrt[]{2}$ o $x \geq - \sqrt[]{2}$
            \end{center}

            Tenemos entonces que $x \geq 1$, $x \leq -1$, $x \leq \sqrt[]{2}$, $x \geq - \sqrt[]{2}$.
            Esto simplificado queda como $\sqrt[]{2} \geq x \geq 1$ o $-1 \leq x \leq \sqrt[]{2}$. 
            
            Por lo que
            $dom f(x)= \{x \in \mathbb{R} | \sqrt[]{2} \geq x \geq 1, -1 \leq x \leq \sqrt[]{2}\}$.

            O también $dom f(x)= \{x \in \mathbb{R} | [\sqrt[]{2}, 1] \bigcup [-1, -\sqrt[]{2}]\}$

            \item $f(x)=\frac{x^2-1}{x+1}$
            
            Para que $\frac{x^2-1}{x+1}$ esté definida, $x+1 \neq 0$ o $x \neq -1$. Por lo que
            $dom f(x)= \{x \in \mathbb{R} | x \neq -1 \} $
        \end{enumerate}
    \item Si $f(x)= \frac{1}{1+x}$, calcule las siguientes expresiones:
        \begin{enumerate}
            \item $f(f(x))$
            
            $f(f(x)) = \frac{1}{1+f(x)}=\frac{1}{1+\frac{1}{1+x}}=\frac{1}{\frac{1+x}{1+x}+\frac{1}{1+x}}=\frac{1+x}{2+x}$\\
            $dom f(f(x))= \{x \in \mathbb{R} | x\neq -1,x\neq -2\}$

            \item $f(\frac{1}{x})$
            
            $f(\frac{1}{x})= \frac{1}{1+\frac{1}{x}}=\frac{1}{\frac{x+1}{x}}=\frac{x}{x+1}$\\
            $dom f(\frac{1}{x})= \{x\in \mathbb{R}| x \neq -1, x \neq 0\}$

            \item $\frac{1}{f(x)}$
            
            $\frac{1}{f(x)}=\frac{1}{\frac{1}{1+x}}=1+x$\\
            $dom \frac{1}{f(x)}= \{x \in \mathbb{R}| x \neq -1\}$

            \item $f(cx)$
            
            $f(cx)= \frac{1}{1+cx}$\\
            $dom f(cx)= \{x \in \mathbb{R}| cx \neq -1\}$

            \item $f(x+y)$
            
            $f(x+y)= \frac{1}{1+(x+y)}= \frac{1}{1+x+y}$\\
            $dom f(x+y)= \{x \in \mathbb{R}| x+y\neq -1\}$

            \item $f(x) + f(y)$
            
            $f(x)+f(y)=\frac{1}{1+x} + \frac{1}{1+y}=\frac{(1+y)+(1+x)}{(1+x)(1+y)}=\frac{x+y+2}{(1+y)+(x+xy)}$\\
            $=\frac{x+y+2}{xy + x + y + 1}$
            $dom f(x)+f(y)= \{x \in \mathbb{R}| x \neq -1, y \neq -1, xy + x + y \neq -1\}$

        \end{enumerate}
    \end{enumerate}
\end{document}
