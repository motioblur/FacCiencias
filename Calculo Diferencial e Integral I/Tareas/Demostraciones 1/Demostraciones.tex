\documentclass[12pt, a4paper]{article}
\usepackage[utf8]{inputenc}
\usepackage{amsmath}
\usepackage{amssymb}
\usepackage{amsthm}
\usepackage{parskip}

\title{Cálculo diferencial e integral I\\Resolución de 'Problemas de: números reales'}
\author{Vite Riveros Carlos Emilio}
\date{20 Agosto 2022}

\begin{document}
\maketitle

3. Pruebe que si $a,b \in \mathbb{R}$, entonces $-(a-b) = b-a$.
    \begin{proof}
        \begin{center}
            \textbf{Demostración:}

            $(b-a)+(a-b)=(b-a)+(a-b)$

            Ley asociativa de la adición (P1):\\
            $(b-a)+(a-b)=b+(-a+a)+(-b)$

            Ley del inverso aditivo (P3):\\
            $(b-a)+(a-b)=b+0+(-b)$

            Ley del neutro aditivo (P2):\\
            $(b-a)+(a-b)=b+(-b)$

            P3:\\
            $(b-a)+(a-b)=0$

            P2:\\
            $(b-a)+(a-b)+(-(a-b))=0+(-(a-b))$

            P3:\\
            $(b-a)+(0)=-(a-b)$

            P2:\\
            $(b-a)=-(a-b)$

        \end{center}
    \end{proof}
        
5. Pruebe que si $a,b \in \mathbb{R}$ son tales que $a^2 = b^2$, entonces $a = b$ o $a = -b$.
    \begin{proof}
        \begin{center}
            \textbf{Demostración:}

            $a^2=b^2$

            $a^2+(ab)=b^2+(ab)$

            Por definición:\\
            $(a*a)+(ab)=(b*b)+(ab)$

            Por ley distributiva del producto (P9):\\
            $a(a+b)=b(b+a)$

            Por ley de conmutación aditiva (P4):\\
            $a(a+b)=b(a+b)$

            Por ley del inverso del producto (P7):\\
            $a(a+b)*(a+b)^-1=b(a+b)*(a+b)^-1$

            $a*1=b*1$

            Por ley del neutro del producto (P6):\\ 
            $a=b$

        \end{center}
    \end{proof}

7. Determine si las siguientes afírmaciones son falsas o verdaderas. Pruebe su respuesta.

(a) Si $a,b \in \mathbb{R}$, entonces $a < a + b$.
    \begin{proof}
        \begin{center}
            \textit{Falso}\\
            \textbf{Demostración:}
    
            $a = 5$ 
            $b = 0$
    
            Sustitución de términos:\\
            $5 < 5 + 0$
    
            (P2):\\
            $5 < 5$
    
        \end{center}
    \end{proof}
(b) Si $a,b \in \mathbb{R}$, entonces $a < a + b$ o $b < a + b$.
    \begin{proof}
        \begin{center}
            \textit{Verdadero}\\
            \textbf{Demostración:}
           
        \end{center}
    \end{proof}
(c) Si $a,b,c,d \in \mathbb{R}$, son tales que $a + c < b + d$, entonces $a < b$ y $c < d$.
    \begin{proof}
        \begin{center}
            \textit{Falso}\\
            \textbf{Demostración:}

            $a = 2$ $b = 1$ $c = 5$ $d = 7$
    
            Sustitución de términos:\\
            $a + c < b + d$\\
            $2+5<1+7$\\
            $7<8$

            $a<b$  $2<1$\\
            y\\
            $c<d$  $5<7$
    
        \end{center}
    \end{proof}
(d) Si $a,b,c,d \in \mathbb{R}$, son tales que $ac < bd$, entonces $a < b$ y $c < d$.
    \begin{proof}
        \begin{center}
            \textit{Falso}\\
            \textbf{Demostración:}

            $a = 8$ $b = 1$ $c = 0$ $d = 3$

            Sustitución de términos:\\
            $ac < bd$\\
            $8*0<1*3$\\
            $0<3$

            $a<b$  $8<1$\\
            y\\
            $c<d$  $0<3$

        \end{center}
    \end{proof}
(e) Si $a,b \in \mathbb{R}$, son tales que $ab = a$, entonces $b = 1$.
    \begin{proof}
        \begin{center}
            \textit{Falso}\\
            \textbf{Demostración:}

            $a = 0$ $b = 9$

            Sustitución de términos:\\
            $ab = a$

            $(0)(9)=0$

            \begin{proof}
                \begin{center}
                    Proposición: Si $a \in \mathbb{R}$, $a*0=0$:\\
                \textbf{Demostración:}

                    $a*0=b$\\
                    $b=a*0$\\

                    (P1):\\
                    $b=a*(a-a)$

                    (P9):\\
                    $b=(a*a)-(a*a)$\\

                    (P1):\\
                    $b=0$\\
                \end{center}
            \end{proof}
            
            $a*0=0$

            $0 = 0$

            $b \neq 1$

        \end{center}
    \end{proof}
(f) Si $a,b \in \mathbb{R}$, son tales que $a^2 \leq b^2$, entonces $a \leq b$.
    \begin{proof}
        \begin{center}
            \textit{Falso}\\
            \textbf{Demostración:}

            $a = 2$ $b = -3$

            Sustitución de términos:\\
            $a^2 \leq b^2$

            $4<9$

            $a<b$  
            
            $2<-3$

        \end{center}
    \end{proof}
9. Pruebe que si $a,b \in \mathbb{R}$, entonces $2ab \leq a^2 + b^2$.
    \begin{proof}
        \begin{center}
            \textbf{Demostración:}

            $(a-2)^2\geq0$

            Por definición de la potenciación:\\
            $(a-2)^2=(a-2)(a-2)$

            (P9):\\
            $(a-2)(a-2)=a(a-b)-b(a-b)$

            $a*a-b*b-b*a+(-b)*(-b)$

            $a*a=a^2$ $b*b=b^2$ $-ab+(-ab)=2ab$

            ${(a-b)}^2=a^2-2ab+b^2$

            $a^2-2ab+b^2\geq0$

            (P1):\\

            $(a^2+b^2)-2ab\geq0$

            $a^2+b^2\geq2ab$

        \end{center}
    \end{proof}
\end{document}
