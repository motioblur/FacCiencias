\documentclass[12pt, a4paper]{article}
\usepackage[utf8]{inputenc}
\usepackage{amsmath}
\usepackage{amssymb}
\usepackage{amsthm}
\usepackage{parskip}

\title{Cálculo diferencial e integral I\\Resolución de "Problemas de: números reales"}
\author{Vite Riveros Carlos Emilio}
\date{26 Agosto 2022}

\begin{document}
    \maketitle
    1. Encuentre todos los números reales que satisfagan las siguientes desigualdades:
    \begin{center}
        (b) $5-x^2<-2$
            
            $-x^2<-2-5$

            $-x^2<-7$

            $x^2>7$

            $\sqrt[]{x^2}>\sqrt[]{7}$

            $|x|>\sqrt[]{7}$

            $x>\sqrt[]{7}$,  $x<-\sqrt[]{7}$
            
        (e) $\frac{1}{x} - \frac{1}{1-x} < 0$ 

            $\frac{1}{x} < \frac{1}{1-x} $

            Si $x \neq 0$, $\frac{1}{x}$ está definido. Entonces:

            $\frac{x}{x} < \frac{1}{1-x}$

            $1 < \frac{x}{x(1-x)}$

            Como $a,b \in \mathbb{R}$

    \end{center}

    5. Pruebe que si $a,b \in \mathbb{R}$, entonces $a^2 \leq b^2$ si y solo si $|a| \leq |b|$.
        \begin{proof}
            \begin{center}
                \textbf{Demostración:}

                \textit{Caso 1:}\\
                $|a|\leq|b|$

                Por definición de el valor absoluto:\\
                $|a|=a$, $|b|=b$

                $a \leq b$

                $a * a \leq b * a$ 
                
                $a * b \leq b * b$

                Por definición de la potenciación:\\

                $a^2 \leq ab$
                
                $ab \leq b^2$

                $a^2 \leq ab \leq b^2 $

                Por transitividad:\\
                $a^2 \leq b^2$

                \textit{Caso 2:}\\
                $a^2 \leq b^2$

                $\sqrt[]{a^2} \leq \sqrt[]{b^2}$

                $|a|\leq|b|$

            \end{center}       
        \end{proof}

    6. En los siguientes incisos, escriba el mismo número quitando (al menos) un signo de valor absoluto.
    \begin{center}
        (b) $||\sqrt[2]{2} + \sqrt[2]{3}| - |\sqrt[2]{5} + \sqrt[2]{7}|| $

        (d) $||a+b| + |c| - |a+b+c|| $

    \end{center}

    7. Calcule todos los números reales que satisfacen las siguientes condiciones:
    \begin{center}
        (c) $|x-1||x+1| < 2$

        (f) $|x-1||x+2| = 3$
    \end{center}
\end{document}